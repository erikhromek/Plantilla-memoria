% Chapter Template

\chapter{Conclusiones} % Main chapter title

\label{Chapter5} % Change X to a consecutive number; for referencing this chapter elsewhere, use \ref{ChapterX}

En este capítulo se presentan las conclusiones del trabajo realizado, así como también propuestas de continuidad para el futuro.
%----------------------------------------------------------------------------------------

%----------------------------------------------------------------------------------------
%	SECTION 1
%----------------------------------------------------------------------------------------

\section{Conclusiones y resultados obtenidos}

En este trabajo se ha culminado el desarrollo y las pruebas funcionales de un prototipo de botón antipánico y un sistema web para poder recibir y gestionar las alertas recibidas.

En relación a los objetivos establecidos, se lograron la mayoría, exceptuando la reducción significativa del consumo energético para prolongar la duración del dispositivo, y la concepción del desarrollo embebido como un proyecto rápidamente extensible, similar a una biblioteca. Se destaca la satisfacción de los demás objetivos, subrayando el valor y contribución del desarrollo realizado. En consecuencia, la etapa final del proyecto se orientó hacia una finalización más integral de los otros componentes involucrados, con miras a su aplicación en futuros proyectos.

El análisis de las conclusiones y resultados generales permite evaluar el cumplimiento de todos los requisitos detallados en el anexo \ref{AppendixA}. Se observa que la mayoría de los problemas encontrados estuvieron relacionados con las dificultades para optimizar el consumo energético. 

Respecto a los objetivos no alcanzados, se atribuye a la subestimación del trabajo necesario en el desarrollo del sistema embebido y la limitada experiencia en este tipo de proyectos. Además, la dificultad para realizar pruebas exhaustivas sobre los componentes de hardware también representó un desafío significativo.

Se considera que la planificación inicial fue adecuada en términos de tiempo, a excepción del módulo embebido, donde se identificaron desafíos y complejidades considerables. Hubo imprevistos laborales que impactaron la gestión del tiempo y llevaron a posponer la finalización del desarrollo.

A pesar de los desafíos enfrentados y considerando tanto los entregables como los resultados obtenidos, se concluye que el trabajo fue altamente satisfactorio, destacándose el aprendizaje, la experiencia adquirida y el desarrollo de un prototipo con potencial para futuras mejoras y aplicaciones.


%----------------------------------------------------------------------------------------
%	SECTION 2
%----------------------------------------------------------------------------------------
\section{Trabajo futuro}

En relación al trabajo futuro, se pueden tener en cuenta dos aristas para aprovechar el trabajo realizado y extenderlo hacia el futuro. A continuación, se detallan los principales puntos que se pueden desarrollar y trabajar hacia adelante.

\begin{itemize}
	\item Tomar como referencia todo lo aplicado para el desarrollo con el framework ESP-IDF, y lo que se investigó pero no se pudo aplicar correctamente e implementarlo para otras placas similares, que tengan algunas características de hardware resueltas o ya incorporadas. Este el caso de la placa de desarrollo \textit{LilyGo SIM7000G}, que integra módulos de telefonía móvil, GPS, batería, etc. en un mismo kit \citep{7600G:1}. Además, incluye la posibilidad de utilizar redes de bajo consumo como NB-IoT \citep{NBIOT:1} ya que posee un módulo GSM más moderno que el utilizado en el trabajo.
	\item Convertir el \textit{Minimun Viable Product} o MVP del sistema web en un producto final o incorporar las funcionalidades consideradas más útiles en un producto \textit{legacy} del empleo actual del autor con características muy similares, como es la parte de \textit{WebSockets}, API REST e integración con números telefónicos virtuales.
	\item Extender la funcionalidad del \textit{frontend} para contemplar múltiples usuarios de una misma organización o ambiente, teniendo ya la implementación en la API.
	\item Incorporar una estructura del \textit{layout} de la aplicación web que sea adaptable a dispositivos móviles.
	\item Incorporar la posibilidad de recibir alertas no solamente mediante SMS, sino también mediante otras aplicaciones como \textit{Whatsapp}, cuya integración puede hacerse vía APIs \citep{TWILIO:3}.
	\item Agregar de funcionalidades de seguimiento continuo en el sistema embebido para vehículos, sabiendo que hay trabajos realizados sobre el tema, pero considerando incorporar alimentación de energía continua.
\end{itemize}
