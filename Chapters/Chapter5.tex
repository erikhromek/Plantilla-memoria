% Chapter Template

\chapter{Conclusiones} % Main chapter title

\label{Chapter5} % Change X to a consecutive number; for referencing this chapter elsewhere, use \ref{ChapterX}


%----------------------------------------------------------------------------------------

%----------------------------------------------------------------------------------------
%	SECTION 1
%----------------------------------------------------------------------------------------

\section{Conclusiones y resultados obtenidos}

En este trabajo, se ha culminado el desarrollo y las pruebas funcionales de un prototipo de botón antipánico y un sistema web para poder recibir y gestionar las alertas recibidas. Originalmente se plantearon los siguientes objetivos:
\begin{itemize}
	\item Desarrollar un prototipo de botón antipánico con un mínimo de características requeridas y un consumo bajo de energía que permita una autonomía de varios días.
	\item Orientar el desarrollo embebido a una biblioteca que pueda ser extendida con más características.
	\item Desarrollar de un sistema web básico que permita gestionar las alertas recibidas.
	\item Aplicar los conocimientos aprendidos durante la cursada de la especialización y tecnologías aledañas investigadas.
\end{itemize}

Además, se buscaron los siguientes objetivos, pero en carácter de objetivos más personales, sin necesariamente influenciar los objetivos primarios del proyecto:
\begin{itemize}
	\item Poder investigar e implementar un servicio \textit{cloud} para contar con números de teléfonos virtuales que permitan recibir mensajes de texto para el reemplazo de aplicaciones \textit{legacy} en el empleo actual del autor.
	\item Desarrollar algunas técnicas como comunicación en tiempo real con \textit{WebSockets} para optimizar características de productos \textit{legacy} del empleo actual del autor.
	\item Concluir la especialización con el desarrollo del trabajo final.
\end{itemize}

En relación a los objetivos alcanzados se considera que el único objetivo que no se alcanzó fue el de poder reducir drásticamente el consumo energético de tal forma que el dispositivo pueda durar varios días, y el hecho de querer plantear el desarrollo embebido como un proyecto que sea rápidamente extensible, como una biblioteca. Por otra parte, los requerimientos menores si se pudieron cumplir, y se considera importante destacarlo, para evidenciar el valor y aporte del desarrollo del trabajo. Debido a esto, se terminó orientando la etapa final del desarrollo a poder completar de una forma más abarcativa los demás componentes involucrados y que puedan aprovecharse para proyectos futuros.

Teniendo las anteriores conclusiones y resultados generales, se puede analizar el grado de cumplimiento de todos los requerimientos en la tabla \ref{tab:requerimientos}

\begin{table}[H]
	\centering
	\caption[\textit{Requerimientos}]{Grado de cumplimiento de requerimientos}
	\begin{tabular}{l c}    
		\toprule
		\textbf{\textit{Requerimiento}} 	 & \textbf{Grado de cumplimiento} \\
		\midrule
		Req \#1.1 & \makecell{No se pudo cumplir de forma satisfactoria al no disminuir \\  el consumo de energía a los valores deseados}    \\		
		Req \#1.2 & Se pudo cumplir    \\	
		Req \#1.3 &   Se pudo cumplir   \\	
		Req \#1.4 &   Se pudo cumplir   \\	
		Req \#1.5 &  Se pudo cumplir    \\	
		Req \#1.6 &  Se pudo cumplir    \\	
		Req \#1.7 &  Se pudo cumplir, habiendo sido modificado   \\	
		Req \#1.8 &  Se pudo cumplir    \\	
		Req \#1.9 &  Se pudo cumplir, habiendo sido ligeramente alterado   \\	
		Req \#1.10 & Se pudo cumplir     \\	
		Req \#1.11 &  Se pudo cumplir con algunas particularidades   \\	
		Req \#2.1 &  Se pudo cumplir    \\	
		Req \#2.2 &   Se pudo cumplir   \\	
		Req \#2.3 & Se pudo cumplir     \\	
		Req \#2.4 &  Se pudo cumplir    \\	
		Req \#2.5 & Se pudo cumplir     \\	
		Req \#2.6 &  Se pudo cumplir    \\	
		Req \#2.7 &  No se cumplió, fue descartado   \\	
		Req \#3.1 &  No se cumplió, fue unificado con el requerimiento \#3.2   \\	
		Req \#3.2 &  \makecell{No se cumplió ya que no se considera al desarrollo embebido \\ apto para justificar la incorporación de un manual}  \\	
		Req \#3.3 &  \makecell{No se cumplió ya que no se considera al desarrollo embebido \\ apto para considerarlo una biblioteca reutilizable}  \\	
		\bottomrule
		\hline
	\end{tabular}
	\label{tab:requerimientos}
\end{table}

Se observa que la mayor cantidad de inconvenientes fue en relación a no poder lograr el objetivo de optimizar el consumo energético.

Por último, se debe hacer un análisis en retrospectiva de los objetivos no cumplidos, ya que en el caso del dispositivo embebido, se considera que solamente el trabajo sobre este componente, hubiese demandado la asignación total del tiempo del trabajo final por diferentes motivos, entre los que se encuentran:
\begin{itemize}
	\item Limitaciones de las características de los módulos para este caso de uso.
	\item Poca experiencia en el desarrollo de sistemas embebidos de este tipo, tanto en el conocimiento teórico como en la ejecución.
	\item Dificultad para realizar pruebas exhaustivas sobre los componentes embebidos.
	\item Poca comprensión sobre particularidades de la implementación de sistemas embebidos, como la incorporación y uso de componentes electrónicos como resistencias, capacitores, reguladores de tensión, entre otros.
\end{itemize}

Se considera que la planificación original estuvo bien en relación al tiempo requerido, a excepción del módulo embebido cuya complejidad e inconvenientes detectados fueron considerables. Respecto a la organización en el tiempo, se considera que no estuvo bien, debido a compromisos laborales y personales que obligaron a posponer la finalización del desarrollo.

De todas maneras, se considera que el resultado del trabajo, teniendo en cuenta no solamente los entregables y resultados sino también el proceso abordado, se considera muy positivo.

%----------------------------------------------------------------------------------------
%	SECTION 2
%----------------------------------------------------------------------------------------
\section{Trabajo futuro}

En relación al trabajo futuro, se pueden tener en cuenta dos aristas para aprovechar el trabajo realizado y extenderlo hacia el futuro. A continuación, se detallan los principales puntos que se pueden desarrollar y trabajar para el proyecto hacia adelante.

\begin{itemize}
	\item Tomar como referencia todo lo aplicado para el desarrollo con el framework ESP-IDF, y lo que se investigó pero no se pudo aplicar correctamente e implementarlo para otras placas similares, que tengan algunas características de hardware resueltas o ya incorporadas. Este el caso de la placa de desarrollo \textit{LilyGo SIM7000G}, una placa que integra ya módulos de telefonía móvil, GPS, batería, etc. en un mismo kit\citep{7600G:1}, incluyendo además la posibilidad de utilizar redes de bajo consumo como NB-IoT\citep{NBIOT:1} ya que posee un módulo GSM más moderno que el utilizado en el trabajo.
	\item Convertir el \textit{Minimun Viable Product} o MVP del sistema web en un producto final o incorporar las funcionalidades consideradas más útiles en un producto \textit{legacy} del empleo actual del autor con características muy similares, como es la parte de \textit{WebSockets}, API REST y integración con números telefónicos virtuales.
	\item Extender la funcionalidad del \textit{frontend} para contemplar múltiples usuarios de una misma organización o ambiente, teniendo ya la implementación en la API.
	\item Incorporar una estructura del \textit{layout} de la aplicación web que sea adaptable a \textit{mobile}.
	\item Incorporar la posibilidad de recibir alertas no solamente mediante SMS, sino también mediante otras aplicaciones como \textit{Whatsapp}, cuya integración puede hacerse vía APIs\citep{TWILIO:3}.
	\item Agregar de funcionalidades de seguimiento continuo en el sistema embebido para vehículos, sabiendo que hay trabajos realizados sobre el tema, pero considerando incorporar alimentación de energía continua.
\end{itemize}
