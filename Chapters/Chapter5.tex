% Chapter Template

\chapter{Conclusiones} % Main chapter title

\label{Chapter5} % Change X to a consecutive number; for referencing this chapter elsewhere, use \ref{ChapterX}

En este capítulo se presentan las conclusiones del trabajo realizado, así como también propuestas de continuidad para el futuro.
%----------------------------------------------------------------------------------------

%----------------------------------------------------------------------------------------
%	SECTION 1
%----------------------------------------------------------------------------------------

\section{Conclusiones y resultados obtenidos}

En este trabajo se ha culminado el desarrollo y las pruebas funcionales de un prototipo de botón antipánico y un sistema web para poder recibir y gestionar las alertas recibidas.

En relación a los objetivos planteados se considera que el único objetivo que no se alcanzó fue el de poder reducir drásticamente el consumo energético de tal forma que el dispositivo pueda durar varios días, y el hecho de querer plantear el desarrollo embebido como un proyecto que sea rápidamente extensible, como una biblioteca. Por otra parte, los demás objetivos sí se pudieron cumplir, y se considera importante destacarlo, para evidenciar el valor y aporte del desarrollo del trabajo. Debido a esto, se orientó la etapa final del desarrollo a poder completar de una forma más abarcativa los demás componentes involucrados para que puedan aprovecharse para proyectos futuros.

En base a las conclusiones y resultados generales, se puede analizar el grado de cumplimiento de todos los requerimientos en el anexo \ref{AppendixA}. Se observa que la mayor cantidad de inconvenientes estuvo relacionada con las dificultades para cumplir con la optimización del consumo energético.

En relación a los objetivos no cumplidos, se debe al hecho de haber subestimado el trabajo a realizar sobre el sistema embebido y la poca experiencia en el desarrollo de sistemas embebidos de este tipo. Otro factor que influyó fue la dificultad para realizar pruebas exhaustivas sobre los componentes de hardware.

Se considera que la planificación original fue acorde al tiempo requerido, a excepción del módulo embebido cuya complejidad e inconvenientes detectados fueron considerables. Respecto a la organización del tiempo, se presentaron imprevistos laborales que obligaron a posponer la finalización del desarrollo.

A pesar de los desafíos atravesados y teniendo en cuenta tanto los entregables como los resultados, se considera que el resultado del trabajo fue muy positivo, destacando el aprendizaje, la experiencia adquirida y el desarrollo de un prototipo con la posibilidad de continuar el trabajo.

%----------------------------------------------------------------------------------------
%	SECTION 2
%----------------------------------------------------------------------------------------
\section{Trabajo futuro}

En relación al trabajo futuro, se pueden tener en cuenta dos aristas para aprovechar el trabajo realizado y extenderlo hacia el futuro. A continuación, se detallan los principales puntos que se pueden desarrollar y trabajar hacia adelante.

\begin{itemize}
	\item Tomar como referencia todo lo aplicado para el desarrollo con el framework ESP-IDF, y lo que se investigó pero no se pudo aplicar correctamente e implementarlo para otras placas similares, que tengan algunas características de hardware resueltas o ya incorporadas. Este el caso de la placa de desarrollo \textit{LilyGo SIM7000G}, que integra módulos de telefonía móvil, GPS, batería, etc. en un mismo kit \citep{7600G:1}. Además, incluye la posibilidad de utilizar redes de bajo consumo como NB-IoT \citep{NBIOT:1} ya que posee un módulo GSM más moderno que el utilizado en el trabajo.
	\item Convertir el \textit{Minimun Viable Product} o MVP del sistema web en un producto final o incorporar las funcionalidades consideradas más útiles en un producto \textit{legacy} del empleo actual del autor con características muy similares, como es la parte de \textit{WebSockets}, API REST e integración con números telefónicos virtuales.
	\item Extender la funcionalidad del \textit{frontend} para contemplar múltiples usuarios de una misma organización o ambiente, teniendo ya la implementación en la API.
	\item Incorporar una estructura del \textit{layout} de la aplicación web que sea adaptable a dispositivos móviles.
	\item Incorporar la posibilidad de recibir alertas no solamente mediante SMS, sino también mediante otras aplicaciones como \textit{Whatsapp}, cuya integración puede hacerse vía APIs \citep{TWILIO:3}.
	\item Agregar de funcionalidades de seguimiento continuo en el sistema embebido para vehículos, sabiendo que hay trabajos realizados sobre el tema, pero considerando incorporar alimentación de energía continua.
\end{itemize}
