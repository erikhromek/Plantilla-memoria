% Appendix A

\chapter{Tabla de requerimientos cumplidos} % Main appendix title

\label{AppendixA} % For referencing this appendix elsewhere, use \ref{AppendixA}

En esta sección se anexa la tabla \ref{tab:requerimientos}, que detalla los requerimientos que se plantearon al inicio del trabajo y el grado de cumplimiento de cada uno.

\begin{table}[H]
	\centering
	\caption{Grado de cumplimiento de requerimientos.}
	\begin{tabular}{l c}    
		\toprule
		\textbf{\textit{Requerimiento}} 	 & \textbf{Grado de cumplimiento} \\
		\midrule
		Req \#1.1 & \makecell{No se pudo cumplir de forma satisfactoria al no disminuir \\  el consumo de energía a los valores deseados.}    \\		
		Req \#1.2 & Se pudo cumplir.    \\	
		Req \#1.3 &   Se pudo cumplir.   \\	
		Req \#1.4 &   Se pudo cumplir.   \\	
		Req \#1.5 &  Se pudo cumplir.    \\	
		Req \#1.6 &  Se pudo cumplir.    \\	
		Req \#1.7 &  Se pudo cumplir, habiendo sido modificado.   \\	
		Req \#1.8 &  Se pudo cumplir.    \\	
		Req \#1.9 &  Se pudo cumplir, habiendo sido ligeramente alterado.   \\	
		Req \#1.10 & Se pudo cumplir.     \\	
		Req \#1.11 &  Se pudo cumplir con algunas particularidades.   \\	
		Req \#2.1 &  Se pudo cumplir.    \\	
		Req \#2.2 &   Se pudo cumplir.   \\	
		Req \#2.3 & Se pudo cumplir.    \\	
		Req \#2.4 &  Se pudo cumplir.    \\	
		Req \#2.5 & Se pudo cumplir.     \\	
		Req \#2.6 &  Se pudo cumplir.    \\	
		Req \#2.7 &  No se cumplió, fue descartado.   \\	
		Req \#3.1 &  No se cumplió, fue unificado con el requerimiento \#3.2.   \\	
		Req \#3.2 &  \makecell{No se cumplió, no se considera al desarrollo embebido \\ apto para justificar la incorporación de un manual.}  \\	
		Req \#3.3 &  \makecell{No se cumplió, no se considera al desarrollo embebido \\ apto para considerarlo una biblioteca reutilizable.}  \\	
		\bottomrule
		\hline
	\end{tabular}
	\label{tab:requerimientos}
\end{table}
